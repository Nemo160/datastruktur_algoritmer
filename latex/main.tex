\documentclass[12pt, a4paper]{article}

% Text efter ett procenttecken är en kommentar
% och kommer inte med i dokumentet.

% Inkludera olika paket för att få
% rätt teckenkodning, rubriker på svenska,
% länkade referenser, med mera.

\usepackage[utf8]{inputenc} % Hantera åäö med UTF-8 teckenkodning. (https://en.wikipedia.org/wiki/UTF-8)


% Ändra de tre raderna nedanv
\usepackage{authblk}

\title{Complexity Analysis}
\author[1]{Edinson Uribe}
\author[2]{Fabian Kåhre}
\affil[1]{\texttt{edinson.uribe01@gmail.com}}
\affil[2]{\texttt{fabian.kahre@gmail.com}}
\date{\today}

% och scrolla sedan ner till \begin{document}
% där dokumentet börjar.


\usepackage[english]{babel}
\usepackage{hyperref}
\usepackage[T1]{fontenc}
\usepackage{caption}
\usepackage{listings}
\usepackage{appendix}
\usepackage{graphicx}
% Visa marginalerna i dokumentet:
%\usepackage[showframe,paper=a4paper]{geometry}

\usepackage{ifxetex}
\ifxetex
   % För KAUs officiella font måste xetex användas
   \usepackage{fontspec}
   \setmainfont{Georgia}
\else
   % Mathdesign finns inte installerat överallt, men om det finns, använd det
   %\usepackage[urw-garamond]{mathdesign}
   % Annars använd Times New Roman som typsnitt. 
   \usepackage{times}
\fi


\lstset{breakatwhitespace=false,
  breaklines=true,
  captionpos=b,
  basicstyle=\ttfamily\small
}


\makeatletter
\renewcommand{\maketitle}{\bgroup\setlength{\parindent}{0pt}
\begin{flushleft}

% KAU logotyp:

\vspace*{-1cm} % flytta upp loggan lite i marginalen
\noindent\includegraphics[width=3cm]{kaulogo.jpg}
\vspace{2cm} % mellanrum mellan logga och titelinformationen

  \hrule\vspace{0.5cm}
  \textbf{\Huge\@title}
  \vspace{0.5cm}\hrule
  
  \vspace{0.5cm}
  \@author
  
  \vspace{0.5cm}
  \@date
  
\vspace{0.5cm}
  DVGB03 Datastrukturer och algoritmer
  
  Datavetenskap
  
  Fakulteten för hälsa-, teknik- och naturvetenskap
  \vspace{0.5cm}\hrule

\end{flushleft}\egroup
}
\makeatother


\begin{document}

\maketitle  % Skriv ut titel, författare, datum

\tableofcontents  % Skriv ut innehållsförteckning

\newpage

\section{Introduction}
\subsection{Background}
bla bla bla \cite{example}
\section{Design and Implementation}
\subsection{Experimental Setup and Results}

\section{Analysis}
\subsection{Problems Encountered}
\section{Conclusion}

\newpage

\bibliography{references}
\bibliographystyle{unsrt}

\begin{appendices}

\section{Program Code}
\label{appendix:code}

% Example using listings:
\begin{lstlisting}[language=Java, caption={Main Sorting Algorithm}]
    // Your code goes here
    public static void sort(int[] arr) {
        ...
    }

\end{lstlisting}

\section{Code References}
\label{apendix:code-references}



\end{appendices}

\end{document}
